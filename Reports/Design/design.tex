\chapter{Design}

% ------------ REQUIREMENTS -------------
\section{System requirements}
\todo[color=yellow]{add small description}

\subsection{Functional requirements}
\begin{table}[H]
\caption{Functional requirements of the software}
\begin{center}
%\begin{tabular}{cc}
\begin{tabular}{| p{0.6cm} | p{6cm} | p{8cm} |}
		\hline
		\textbf{No.} & \textbf{Requirements} & \textbf{Description} \\ \hline
    \multicolumn{3}{| l |}{\textbf{Menu}}\\ \hline
    1 & Shows the list of playable algorithms & In the menu, the program is to show all the algorithms available in the program in the main list. In this list, user can select whichever algorithm they wish to see. \\ \hline
    2 & Classify the available algorithms between the 3 main algorithmic paradigms & On the main list, the algorithms are to be classified between the 3 main paradigms, such as the greedy method, divide and conquer, and dynamic programming. This is to allow the users to understand immediately the correlation between similar algorithms when classified within its paradigms. This is also to increase the ease of usability, as users will only be required to look within the algorithms paradigm to search for a specific problem.  \\ \hline
    \multicolumn{3}{| l |}{\textbf{Animation}}\\ \hline
    3 & Plays the animation & When the animation is in its initial or paused state, users can play the animation. This initiates the animation, which plays until the end, unless the user either pauses or stops the animation. \\ \hline
    4 & Pauses the animation & The user can pause the animation, which stops the animation temporarily at its current state. \\ \hline
    5 & Stops the animation & When the animation is playing, user can stop the animation. This ends the animation completely at any point of time during the playtime of the animation. \\ \hline
    6 & Backtracks the animation & During the animation's playtime, the program keeps track on the number of iteration(s) the animation is currently at. When a user chooses to backtrack the animation, the animation will \textit{rewind} itself from its current iteration \textit{i}, to \textit{i - 1}. \\ \hline
    7 & Shows a short description during the animation on each \textit{iteration} of the algorithm & During the animation's playtime, the program is to show a short description about what the animation is doing. \\ \hline
\end{tabular}
\end{center}
\label{tab:multicol}
\end{table}

\newpage
   
\begin{table}[ht]
\caption{Functional requirements of the software}
\begin{center}
%\begin{tabular}{cc}
\begin{tabular}{| p{0.6cm} | p{6cm} | p{8cm} |}
		\hline
		\textbf{No.} & \textbf{Requirements} & \textbf{Description} \\ \hline
    \multicolumn{3}{| l |}{\textbf{Help option}}\\ \hline
    8 & Adjust the speed of the animation & Users can adjust the speed of the animation ranging from 1 (very slow), to 10 (very fast). By default, the speed of the animation will be set to 5. \\ \hline
    9 & Adjust the font size & Users can adjust the font size to fit their own requirements. Users can pick sizes from small (font size 8), default (font size 12), and large (font size 16). By default, the general size of the fonts in the program will be sized 12. \\ \hline
    \multicolumn{3}{| l |}{\textbf{Additional features}}\\ \hline
    10 & Suggests to play similar algorithms & When users view a certain algorithm, the program also suggests an algorithm alike with the currently viewed one. This is to enhance better learning experience for users to seek out on similar problems \\ \hline
    11 & Appendix that shows further writeup of the algorithms available in the program & This shows the full writeup of the description shown during the animation, and additional information in regards with the algorithm. \\ \hline
\end{tabular}
\end{center}
\label{tab:multicol}
\end{table}

\newpage

\subsection{Non-functional requirements}
\todo[color=yellow]{Not sure if saved settings belong in non-functional}
\begin{table}[ht]
\caption{Non-functional requirements of the software}
\begin{center}
%\begin{tabular}{cc}
\begin{tabular}{| p{0.6cm} | p{6cm} | p{8cm} |}
		\hline
		\textbf{No.} & \textbf{Requirements} & \textbf{Description} \\ \hline
    \multicolumn{3}{| l |}{\textbf{Graphical interface}}\\ \hline
    1 & The images for the animation is to be scalable depending on the size of the user's input & The physical size of the animation highly depends on the input size given by either the user or the random generator. Due to this, the program needs to carefully scale the animation when it is either too small or too big for the screen. It needs to ensure that the user can easily see the images and fonts of the animation, whether the input size is small or large. \\ \hline
    2 & Tables included in the animation demonstration are to be scrollable when it gets larger than a specified size given & Some algorithms require a table, especially the dynamic programming types. The table varies in size depending on the size of input for the algorithm. If the table width and length gets bigger than a specific size given, instead of exceeding the size, the program is to add a scrollable feature for the table. \\ \hline
    3 & The program is to be clear and easy enough for users to comprehend its design & The colour scheme of the program is to have a calming, non-blaring proposition. The images and fonts along with it needs to be shown clearly, and easily relatable for the general public. \\ \hline
    \multicolumn{3}{| l |}{\textbf{Settings}}\\ \hline
    4 & Saves the settings provided by user & The program is to save the changes made by user under settings. This means that when the user opens the program again, the changed settings will still be in placed. \\ \hline
\end{tabular}
\end{center}
\label{tab:multicol}
\end{table}

% ---------- UML CASE DIAGRAM ----------
\section{UML case diagram}
The use case diagram below on figure \ref{useCaseDiagram} is the representation of what the user can do to interact with the system represented in use cases. It is basically shows the relationship between the user and the system, for this case the student's interaction with the algorithm animation program.

% UML case diagram
\begin{figure}[H]
\centering
\hspace*{-1cm}
\includegraphics[scale=1]{images/report_images/useCaseDiagram.png}
\caption{The system use case diagram}
\label{useCaseDiagram}
\end{figure}

According to the use case diagram on figure \ref{useCaseDiagram}, firstly, the user can change the settings in the program, by changing features such as the speed of the animation, or the font size displayed in the program. This allows the user to work within the environment that is most comfortable for them. 

Other than changing the settings, the user can also select an algorithm which they wish to learn. This will lead them to the page where the animation of the algorithm is. From here, the user can manipulate the animation, by pressing controls such as play, pause, stop and ``backtrack''. For more information in regards to these controls, refer to the \todo[color=orange]{add some glossary or some sort}. From this page, the user can also access the algorithms which are \textit{related} to the one in question, if they ever wish to do so. This will lead to the page of the algorithm along with its animation.

Other than the settings and the animated feature, if the user ever wishes to know more about the algorithms, the user can view them in the appendix within the program. In the appendix, the user will find all the algorithms available in the program in a list. Once the user selects a particular one they wish to see, the page will display a fuller information in regards to the algorithm. This includes the written up information, and might also involve a few images as well.

% --------------- FLOWCHART -------------------
\section{System flowcharts}
In this section, I have included the flowcharts of the program, to display the how the program flows in general, and how decisions controls its following events. 

\subsection{Main flow of the program}
The flowchart below on figure \ref{mainFlowChart} shows the overall flow of the program itself. The three main modules which makes up the program, i.e. change settings, view the animation of algorithms, and the appendix, are grouped in its respective subprocesses. Each subprocess will be described more in detail in sections \ref{sec:flowchartAnimation}, for the animation module, and \ref{} for the appendix module.

\begin{figure}[H]
\centering
\hspace*{-1cm}
\includegraphics[scale=1]{images/report_images/flowchartMain.png}
\caption{The flowchart of the whole system.}
\label{mainFlowChart}
\end{figure}

On the main flow chart below, the user first starts the application, which then brings them to the main menu. From the main menu, the user can select up to three features they wish to use, which are \textit{Select}, \textit{List of Algorithms}, and \textit{Appendix}. Doing so will bring to the respective features. From the main menu, the user can also choose to exit from the main menu, by selecting \textit{Exit} which then closes the whole application.

\subsection{Flow of the animation module} \label{sec:flowchartAnimation}
The animation module is admittedly the main feature of the program that will be heavily concentrated during the course of the implementation of the project. The user first selects the \textit{List of Algorithms} button, which then leads to the animation module. From here, there will be a list of the algorithms classified between 3 main paradigms and a sorting algorithm section. When a user selects an algorithm they wish to view, this will lead them to a page where the animation (in stopped mode), and the small description in regards to the algorithm. 

The program will first prompt the user either to enter their own specific input, or the generate a random value instead. When the user decides to add their own input, there will be a specific limit assigned to the algorithm. If the input exceeds the limited amount, the program will throw an error message to inform user that the input was unacceptable and requests the user to add an input that does not exceed the assigned limit. On the other hand, if the user selects to generate random value, the program will generate a random value within the limited amount assigned.

Once an input has been either retrieved or generated, the user then will able to play the animation by pressing the play button. Whilst the animation is at its \textit{playing state}, the user can control the animation by either \textit{pause}, \textit{backtrack}, or \textit{stop} the animation. The state of the animation depends on what type of controls have been selected by the user, and below is a table \ref{tab:animationControls}, shows the outcome of the animation's state when a particular control button has been selected by the user.

\begin{table}[H]
\caption{The list of animation controls.}
\begin{center}
%\begin{tabular}{cc}
\begin{tabular}{| p{4cm} | p{11cm} |}
		\hline
		\textbf{Control} & \textbf{Description} \\ \hline
		Play & This button simply initiates the animation. Only available when the animation is either at its initial stage, \textit{paused}, or \textit{stopped}. \\ \hline
     Pause & When a paused button is activated whilst the animation is playing, the animation stops temporarily. The stopped time will be saved, and will continue from that time if whenever the user chooses to play the animation. User can only pause the animation when the animation is being played. \\ \hline
     Backtrack &  This is a unique feature that comes in with the program. As the animation is animated through the use of \textit{iterations}, these iteration values will be counted and stored programmatically. When a user clicks backtrack, the iteration counter, \textit{i}, will be brought back to the previous iteration, which is \textit{i - 1}. Once it goes back to its previous iteration, the animation will be brought to its \textit{paused} state. From here, the user can press \textit{play}, which will initiate the animation from that particular state. \\ \hline
     Stops &  A stopped button will completely halt the process of the animation. Its final playing state will be discarded once a stop button is selected. \\ \hline
\end{tabular}
\end{center}
\label{tab:animationControls}
\end{table}

\begin{landscape}
\begin{figure}[H]
\centering
\hspace*{-2cm}
\includegraphics[scale=0.9]{images/report_images/flowchartAnimation.png}
\caption{The flowchart of the animation module.}
\label{flowchartAnimation}
\end{figure}
\end{landscape}

\newpage

\subsection{Flow of the appendix module}
Finally, the last module would be the appendix module, which will contain the supplementary material in regards to the algorithms that are used in this program. This basically lists all the algorithms that are used, and users can select any algorithm within that list to view more information about the algorithm.

\begin{figure}[H]
\centering
%\hspace*{-1.5cm}
\includegraphics[scale=1]{images/report_images/flowchartAppendix.png}
\caption{The flowchart of the appendix module.}
\label{flowchartAppendix}
\end{figure}

The flowchart on figure \ref{flowchartAppendix} refers to the sequence of events that are involved within the module. First, from the main menu, as when the user selects \textit{View appendix}, the program then brings the user to the list of all the available appendix found in the program. The user then could select the algorithm they wish to find out more about, by clicking into one. This will then lead the user to the page that predominantly presents the detailed information in regards to the algorithm in question.

% --------------- UI DESIGN -------------------
\newpage
\section{Graphical User Interface design of the system}
\subsection{Program start page}

\begin{figure}[H]
\centering
%\hspace*{-1.5cm}
\includegraphics[scale=1]{images/report_images/uiStartWindow.png}
\caption{The start page of the program}
\label{uiStartWindow}
\end{figure}

As the program first initiates, the UI design shown in figure \ref{uiStartWindow} will be the start page of the application. The start page displays the main menu of the application, which as mentioned earlier in sections \ref{}. 

\begin{enumerate}
\item The \textit{List of Algorithm} button leads the user to the list of algorithms. In this list, user can select whatever algorithm they wish to learn. Refer to section \ref{}, figure \ref{} for more details.
\item The \textit{Change settings} button on the other hand, leads the user to a settings page.
\item The \textit{View appendix} page brings the user to the main appendix list, which lists all the algorithms that is shown within the program.
\item Finally, the \textit{Exit} button would close the whole application, if the user selects it.
\end{enumerate}

\newpage

\subsection{Settings page}

\begin{figure}[H]
\centering
\hspace*{-0.5cm}
\includegraphics[scale=0.8]{images/report_images/uiSettings.png}
\caption{The settings page.}
\label{uiSettings}
\end{figure}

On the \textit{Settings} page, users can adjust several features of the program, for this case, the font size, and the animation speed. The reason for having a settings page is to ensure that the users will have the ease in working in a comfortable environment.

\begin{enumerate}
\item One of the features the user can change is the font size. In order to change the font size, the user is to use the slider below. From the leftmost part of the slider is the smallest size of the font, which is size 9pt. The default size on the other hand is 12pt, followed by the largest possible size is 18pt. The user can adjust the font size from 9pt (smallest) to 18pt (biggest) by using the slider.
\item Secondly, the user can also change the animation speed. Initially, the animation will be running on a default speed of \todo[color=orange]{find the specific speed of the animation!}. However, if the speed is either too fast or slow for the user, the user could always adjust the speed by sliding to the leftmost bit of the slider for slower speed, and rightmost for a faster speed.
\end{enumerate}

\subsection{List of algorithms page}

\begin{figure}[H]
\centering
\hspace*{-0.5cm}
\includegraphics[scale=0.8]{images/report_images/uiListOfAlgorithms.png}
\caption{The page that shows the list of algorithms available for animation.}
\label{uiListOfAlgorithms}
\end{figure}

In this page basically shows the list of all the algorithms that are available in animation. These algorithms are also to be classified between the three main algorithmic paradigms, which are \textit{greedy method}, \textit{divide and conquer}, and \textit{dynamic programming} approaches. Also, another classification would be the sorting algorithms will be included in the list as well. \textit{Note that the size of program has been extended in order to fit all the whole list of algorithms within one screenshot.}

\begin{enumerate}
\item Shows the list of all the algorithms available. These algorithms are also classified according to its respective algorithmic paradigms. Users can click any of those algorithms displayed in that list if they wish to learn more about them.
\end{enumerate}

\subsection{The page of the animation of the algorithm selected}

\begin{figure}[H]
\centering
\hspace*{-0.5cm}
\includegraphics[scale=0.7]{images/report_images/uiInputAnimation.png}
\caption{The page that shows requests the input from the user before starting the animation of the algorithm.}
\label{uiInputAnimation}
\end{figure}

\begin{enumerate}
\item The title of the algorithm in question.
\item The form page of the list of input that is required for the heap sort animation.
\item As all fields are equipped with validation, it will throw an error message in regards to the field if it happens to receive the wrong input. For this case, the user fails to insert a numerical value.
\item Another option instead of adding the user's own input is to generate random values for the animation.
\item This button submits the values and proceeds to the next page, shown in figure \ref{uiAnimation}.
\item This button simply brings the user back to the list of algorithms.
\item This dialog will be prompted once the user clicks the \textit{Submit} button. This is to ensure that the user is happy with the input they have given. From here, if they click \textit{No, make further changes}, the dialog will close, and the page remains the same. If the user clicks \textit{Yes, proceed}, the program will proceed to the next page, which is shown in figure \ref{uiAnimation}.
\end{enumerate}

\begin{figure}[H]
\centering
\hspace*{-0.5cm}
\includegraphics[scale=0.8]{images/report_images/uiAnimation.png}
\caption{The page that shows the animation of the algorithm.}
\label{uiAnimation}
\end{figure}

\begin{enumerate}
\item The title of the algorithm in question.
\item The section of the page where the animation of the algorithm is carried out.
\item The backtrack button.
\item The pause button.
\item The play button. Turns red when the animation is at the \textit{playing state}. This applies to other control buttons as well.
\item The stop button.
\item The number of iteration the animation is currently at. Every time the animation finishes its \textit{main loop}, the iteration counter is added, and it will be displayed here.
\item The area where a written information regarding the animation is displayed. Every time an animation displays something new, a new text block is displayed here, along with the numbers (variables) involved the animation. Once the animation has moved on to the next step, the text mentioned will be greyed out, as the new text will be the one that is emphasized. 
\item The button simply brings the user back to the previous page which would be the main list of algorithms. Whatever that is played in the animation will be discarded.
\end{enumerate}

