\chapter{Requirements}

\begin{table}[ht]
\caption{Functional requirements of the software}
\begin{center}
%\begin{tabular}{cc}
\begin{tabular}{| p{0.6cm} | p{6cm} | p{8cm} |}
		\hline
		\textbf{No.} & \textbf{Requirements} & \textbf{Description} \\ \hline
    \multicolumn{3}{| l |}{\textbf{Menu}}\\ \hline
    1 & Shows the list of playable algorithms & In the menu, the program is to show all the algorithms available in the program in the main list. In this list, user can select whichever algorithm they wish to see. \\ \hline
    2 & Classify the available algorithms between the 3 main algorithmic paradigms & On the main list, the algorithms are to be classified between the 3 main paradigms, such as the greedy method, divide and conquer, and dynamic programming. This is to allow the users to understand immediately the correlation between similar algorithms when classified within its paradigms. This is also to increase the ease of usability, as users will only be required to look within the algorithms paradigm to search for a specific problem.  \\ \hline
    \multicolumn{3}{| l |}{\textbf{Animation}}\\ \hline
    3 & Plays the animation & When the animation is in its initial or paused state, users can play the animation. This initiates the animation, which plays until the end, unless the user either pauses or stops the animation. \\ \hline
    4 & Pauses the animation & The user can pause the animation, which stops the animation temporarily at its current state. \\ \hline
    5 & Stops the animation & When the animation is playing, user can stop the animation. This ends the animation completely at any point of time during the playtime of the animation. \\ \hline
    6 & Backtracks the animation & During the animation's playtime, the program keeps track on the number of iteration(s) the animation is currently at. When a user chooses to backtrack the animation, the animation will \textit{rewind} itself from its current iteration \textit{i}, to \textit{i - 1}. \\ \hline
    7 & Shows a short description during the animation on each \textit{iteration} of the algorithm & During the animation's playtime, the program is to show a short description about what the animation is doing. \\ \hline
\end{tabular}
\end{center}
\label{tab:multicol}
\end{table}

\newpage
   
\begin{table}[ht]
\caption{Functional requirements of the software}
\begin{center}
%\begin{tabular}{cc}
\begin{tabular}{| p{0.6cm} | p{6cm} | p{8cm} |}
		\hline
		\textbf{No.} & \textbf{Requirements} & \textbf{Description} \\ \hline
    \multicolumn{3}{| l |}{\textbf{Help option}}\\ \hline
    8 & Adjust the speed of the animation & Users can adjust the speed of the animation ranging from 1 (very slow), to 10 (very fast). By default, the speed of the animation will be set to 5. \\ \hline
    9 & Adjust the font size & Users can adjust the font size to fit their own requirements. Users can pick sizes from small (font size 8), default (font size 12), and large (font size 16). By default, the general size of the fonts in the program will be sized 12. \\ \hline
    \multicolumn{3}{| l |}{\textbf{Additional features}}\\ \hline
    10 & Suggests to play similar algorithms & When users view a certain algorithm, the program also suggests an algorithm alike with the currently viewed one. This is to enhance better learning experience for users to seek out on similar problems \\ \hline
    11 & Appendix that shows further writeup of the algorithms available in the program & This shows the full writeup of the description shown during the animation, and additional information in regards with the algorithm. \\ \hline
\end{tabular}
\end{center}
\label{tab:multicol}
\end{table}

\newpage

\todo[color=yellow]{Not sure if saved settings belong in non-functional}

\begin{table}[ht]
\caption{Non-functional requirements of the software}
\begin{center}
%\begin{tabular}{cc}
\begin{tabular}{| p{0.6cm} | p{6cm} | p{8cm} |}
		\hline
		\textbf{No.} & \textbf{Requirements} & \textbf{Description} \\ \hline
    \multicolumn{3}{| l |}{\textbf{Graphical interface}}\\ \hline
    1 & The images for the animation is to be scalable depending on the size of the user's input & The physical size of the animation highly depends on the input size given by either the user or the random generator. Due to this, the program needs to carefully scale the animation when it is either too small or too big for the screen. It needs to ensure that the user can easily see the images and fonts of the animation, whether the input size is small or large. \\ \hline
    2 & Tables included in the animation demonstration are to be scrollable when it gets larger than a specified size given & Some algorithms require a table, especially the dynamic programming types. The table varies in size depending on the size of input for the algorithm. If the table width and length gets bigger than a specific size given, instead of exceeding the size, the program is to add a scrollable feature for the table. \\ \hline
    3 & The program is to be clear and easy enough for users to comprehend its design & The colour scheme of the program is to have a calming, non-blaring proposition. The images and fonts along with it needs to be shown clearly, and easily relatable for the general public. \\ \hline
    \multicolumn{3}{| l |}{\textbf{Settings}}\\ \hline
    4 & Saves the settings provided by user & The program is to save the changes made by user under settings. This means that when the user opens the program again, the changed settings will still be in placed. \\ \hline
\end{tabular}
\end{center}
\label{tab:multicol}
\end{table}

