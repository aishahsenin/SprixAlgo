\chapter{Overview}

\section{Project description}
This project primarily focuses on the animation of different types of commonly used algorithms, for the benefit of users to further understand how algorithms work in general. The scope of this project is within the animation of the main algorithmic paradigms, divide and conquer, greedy method, and dynamic programming. 

Learning about what algorithms are and how they work is essential for students who are studying computer science. Since this project is meant to be educational, the target audience of the software will be students studying computer science, or at least have an interest on how computer programs are made efficient.

This project is to develop a software that displays animations that shows how an algorithmic solution works in general. From the program, the users are able to pick the algorithmic solution they wish to learn, enter a certain amount of input, or generate random values, and then learn how the algorithm works by watching the animations presented to them.

\section{Aims and objectives of this project}
% obj 1 - make difficult algorithms easier to understand
The primary objective to this project is simply to make difficult algorithms easily understood. Also, using of visual aid as part of the educational process, for instance animations, to enhance the users' learning experience, which will make the students to learn new algorithmic problems with convenience and ease.

% aim 1 - achieve greater understanding about algorithms
It is generally known the algorithms is one of challenging topics within the computer science field that is difficult for students to grasp on. So, the aim for this project is to allow students to achieve greater understanding in algorithmic paradigms, by providing an animated explanation in a step by step basis. To achieve this, the animation is to allow further speculation on how it works step by step, by breaking it down into smaller parts. This strategy of scrutinizing the algorithm allows the users to speculate the complicated algorithms in its granulated state, on how it works in each step, and then making a connection between the sequence of steps that makes the algorithm work as a whole.

% aim 2 - allow to be the platform to work from for further iterations
Another aim for this project, is to provide the basic idea of how an educational program is suppose to look and work like in order to successfully assist the students. As a computer science student myself, I understand what are the specific difficulties when it comes to learning algorithms, and using them to address every difficulty I had when learning algorithms for the first time. Hopefully, once this project is completed, it will show the other developers who are interested in taking on this project on the specific areas to pay attention to when developing an educational program like this one. 

I have also intend to serve this program as a base, where other developers could use to iterate from, by populating the list of available algorithms, by adding other algorithms into the list. If the project is deemed successful, universities could use this program to assist other students who are studying algorithms, or have difficulty understanding the concept of them.

% aim 3 - increase students' interests on the topic
Another aim for this project that would be nice to achieve, other than to benefit the students learning process, is to increase the students' interest on this topic. Algorithms is one of my favourite topics I had came across as a computer science student during my course in university. By designing and developing this program, I hope to achieve the same sentiments in regards to my interest in algorithms to other students who are studying this topic as well.

\section{Summary of research and analysis}
\todo[color=yellow]{Do this}