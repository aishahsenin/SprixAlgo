\chapter{Overview}

\section{Project description}
This project primarily focuses on the animation of different types of commonly used algorithms, for the benefit of users to further understand how algorithms work in general. Learning about what algorithms are and how they work is essential for students who are studying computer science. Since this project is meant to be educational, the target audience of the software will be students studying computer science, or at least have an interest on how computer programs are made efficient. In this project, the scope will revolve around the animations within the three main algorithmic paradigms, such as the greedy method, divide and conquer, and dynamic programming. Also, some of the sorting algorithms as well.

The primary purpose of this project is to develop a software that displays animations which shows how different types of algorithms within the 3 main paradigms and sorting, works in general. From the program, the users are able to pick the algorithmic solution they wish to learn, either enter a certain amount of input or generate random values, and then learn how the algorithm works by watching the animations presented to them.

\section{Aims and objectives of this project}
% obj 1 - make difficult algorithms easier to understand
The primary objective to this project is simply to make difficult algorithms easily understood. To achieve such feat for instance, would be using animations, which acts as a visual aid for the students to learn the algorithms. As a computer science student myself, I believe that using visual tools such as the animations, would certainly enhance the students' learning experience in regards to this topic. This would also allow students to learn new algorithmic problems with convenience and ease.

% aim 1 - achieve greater understanding about algorithms
It is generally known that algorithms is one of the more challenging topics within the computer science field, and is difficult for students to grasp at times. In this case, another aim for this project is to make this software as an additional tool, that could be used outside lecture periods, and assist students to achieve greater understanding in algorithmic paradigms. To achieve this, the program is to provide a comprehensive animated explanation of the algorithms in a step by step basis. 
%By doing this, it would allow the students to conduct further speculation on how those algorithms work step by step, by breaking it down into smaller parts. 
This strategy of scrutinizing the algorithm allows the users to speculate the complicated algorithms in its granulated state, on how it works in each step, and then making a connection between the sequence of steps that makes the algorithm work as a whole.

% aim 2 - allow to be the platform to work from for further iterations
Another aim for this project, is to provide the basic idea of how an educational program is suppose to look and work like in order to successfully assist the students. As a computer science student myself, I understand what are the specific difficulties when it comes to learning algorithms, and using them to address every difficulty I had when learning algorithms for the first time. Hopefully, once this project is completed, it will show the other developers who are interested in taking on this project on, the specific areas to pay attention to when developing an educational program like this one. 

One of my intentions for this project is to serve this software as foundation. From here, other developers who could use this as a platform, and populate the existing list by many other algorithms and its animations. If the project is deemed successful, universities could use this software to assist other computer science students who are studying algorithms, or generally facing difficulty understanding the concept of them.

% aim 3 - increase students' interests on the topic
Another aim for this project that would be nice to achieve, other than to benefit the students learning process, is to increase the students' interest on this topic. Algorithms is one of my favourite topics I had came across as a computer science student during my course in university. By designing and developing this program, I hope to achieve the same sentiments in regards to my interest in algorithms to other students who are studying this topic as well.

\section{Summary of research and analysis}

During the design phase, I have made some research on the algorithms that I found feasible to apply in this project, with the time that I was given before completion. The algorithms I have chosen are also the ones that I have understood quite intensively, and will certainly be applied in code during the implementation stage. My only concern when it comes to this project, other than the amount of work that I have proposed in the design documentation with the time that I has been allocated to, is implementing the algorithms in animation. Fortunately, after much research in regards of \textit{C\#} and with the help with Microsoft Visual Studio Blend 2015, I believe that I will able to reach the goal in interpreting my knowledge in algorithms into animations.

In spite of this, the design documentation mainly consists of the structure of the program, such as its graphical interface, what are the components that will be included, and generally what is expected overall from the \textit{Algorithm Animation Program}. Whatever research and plans that I have conducted during the design specification are all documented here. Of course, if everything goes well, these plans will be carried forward towards the implementation phase, where the components are finally put into place and put into actual use. 
