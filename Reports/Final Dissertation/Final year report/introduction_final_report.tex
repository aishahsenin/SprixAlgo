\chapter{Introduction}

\section{Project Description}
This project primarily focuses on the animation of different types of commonly used algorithms. The benefit this, is to let users to understand how algorithms work in general. Learning about what algorithms are and how they work is essential for students who are studying computer science, and those who are interested in understanding on designing and building the architecture of an efficient software.  Since this project is meant to be educational, the target audience of the software will be students studying computer science, or at least have an interest on how software are made efficient. In this project, the scope will revolve around the animations within the three main algorithmic paradigms, such as the greedy method, divide and conquer, and dynamic programming. Of course, some of the sorting algorithms will be included as well, as these types of algorithms are considered the essence, which means that it needs to be understood in order to understand wholly on the purpose of algorithms.

The primary purpose of this project is to develop a software that displays animations in which shows different types of algorithms within the three main paradigms and sorting works in general. From the program, the users are able to select an algorithm they wish to learn. From here, the user can choose, within a designated range, can either enter their own set of inputs, or generate a random set of inputs for the algorithm to take in. After which, the algorithm will use these inputs and work accordingly by displaying the animation of the algorithm selected on the program.

\section{Project Aims}
% aim 1 - to provide a clear representation of how algorithms work with the use of animations
The primary objective to this project is simply to make difficult algorithms to be easily understood. To achieve such feat, there would be a need to use animations. These animations act as a visual aids for the students to learn the algorithms from a different perception. As a computer science student myself, I believe that using visual tools such as the animations would certainly enhance the students' learning experience. 

% aim 2 - 
In is generally known that algorithms is one of the challenging topics within the computer science field, and is difficult for some students to grasp at times. Making this another aim of the project is to build this product as a tool for assistance, where students use the program to help them understand further about the project.

\section{Project Objectives}
% obj 1
% obj 2
% obj 3

\section{Project Challenges}
% cha 1 - finding the suitable technology and platform for the construction of the animations
% cha 2 - planning and the construction of the animations 
% cha 3 - testing of the animations with the use of customisable data

\section{Solutions}
% cha 1 - the use of web applications
% cha 2 - 
% cha 3 - 

\section{Effectiveness and Success of the Project}